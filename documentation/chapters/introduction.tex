\section{Einführung}
Einführung in die Thematik. Aufbau der Applikation, Imports (Gurobi, armadillo und co)
\subsection{Verwendung von C++}
Als Programmiersprache wurde in diesem Projekt C++ verwendet. Die entscheidenden Gründe für C++ waren zum einen die hohe Performance sowie die existierenden Bibliotheken (z.B. Armadillo) und Schnittstellen für die Nutzung von Gurobi.  
\subsection{Armadillo}
\href{http://arma.sourceforge.net/}{Armadillo} ist eine C++-Bibliothek, welche Funktionen und Klassen zur linearen Algebra bereitstellt. Angelehnt an MatLab werden hier Klassen für Vektoren, Matrizen und Kuben bereitgestellt.
Im Rahmen dieses Projektes wurde Armadillo in der neusten stabilen Version (8.500.1) verwendet. \\
\subsection{Gurobi}
Mit dem Gurobi Solver wurde im Rahmen dieser Praktikumsaufgabe eine externe Software genutzt um die Optimierungsprobleme zu lösen. Eine Schnittstelle zur Ansteuerung des Gurobi-Solvers in der Version 8.5.0 wird hier vom Hersteller bereitgestellt. 
\subsection{LAPACK}
Als Abhängigkeit von Armadillo wurde außerdem LAPACK genutzt. Um die komplexeren Berechnungen, welche mit Armadillo durchgeführt werden können, zu beschleunigen, wird unter Windows OpenBLAS (oder Intel MK) benötigt. Die Bibliothek LAPACK stellt hierfür ebenfalls Komponenten bereit.