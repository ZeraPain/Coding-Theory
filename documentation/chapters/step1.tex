\section{Schritt 1}
Ziel von Schritt 1 war es, anhand der Werte für k und q, eine Generatormatrix zur konstruieren. \\
Die Anwendung nimmt zunächst die Werte  entgegen und berechnet eine Sammlung von linear unabhängigen Vektoren.\\
Sind alle linear voneinander unabhängigen Vektoren gefunden, wird anhand der Werte von k und q die Größe der Matrix berechnet. Die Berechnung basiert auf der, in der Vorlesung vorgestellten, Formel: $\dfrac{q^{k}-1}{q-1}$. \\
Anhand der berechneten Größe der Matrix, wird zum Abschluss eine Generatormatrix aus den zuvor erstellten Vektoren generiert und ausgegeben.
\subsection{Generierung von linear unabhängigen Vektoren}
\begin{lstlisting}[caption=Generierung der Vektoren]
for (auto k_index = k - 1; k_index >= 0; --k_index)
{
	vec init_vec(k); // set size to k
	init_vec.fill(0); // init with zero
	init_vec.at(k_index) = 1;
	ret.push_back(init_vec);
	calc_lin_independ_vec(ret, &init_vec, 1, q, k_index, k);
}	
\end{lstlisting}
\begin{lstlisting}[caption=Berechnung der linearen Unabhängigkeiten]
k_index++;
auto construct_vec = *init_vec;
calc_lin_independ_vec(ret, &construct_vec, q_index, q, k_index, k);
	
while (construct_vec.at(k_index) < (q - 1))
{
	construct_vec.at(k_index) += 1;
	calc_lin_independ_vec(ret, &construct_vec, q_index + 1, q, k_index, k);
}
\end{lstlisting}

\subsection{Erstellen der Matrix}
\begin{lstlisting}[caption=Erstellen der Generatormatrix]
arma::mat A(mat_size, mat_size);
for (auto i = 0; i < mat_size; ++i)
{
	for (auto j = 0; j < mat_size; ++j)
	{
		const auto res = static_cast<int>(dot(lin_independ_vecs.at(i), lin_independ_vecs.at(j))) % q;
		A(i, j) = 0 == res ? 1 : 0;
	}
}
\end{lstlisting}
  
\subsection{Tests}
Getestet wurde dieser Teil der Anwendung u.a. für die folgenden Wertepaare:

\begin{itemize}
	\item bli
	\item bla
	\item blubb
	
\end{itemize}